\documentclass[a4paper,12pt]{report}

\begin{document}

\title{Models in Pixel Based Phenology}
\author{Joel Granados \\ jogr@itu.dk}
\date{ \today }

\maketitle

\section*{Introduction}
Phenology is the study of plant and animal cycles and how they relate to the
seasons. Phenological measurements are of great importance to ecologists for a
number of reasons. We are interested not in the direct consequence of
phenological measurements, but in the way these measurements are done. More
specifically, we are studying a way to use digital imaging to automate plant
phenology data gathering.

\section{General Model: Naive Bayes Model}

\section{Description of the Data}
The data that we are handling in this experiment comes from a pan tilt zoom
camera. This camera took pictures of a region of interest. This was done
throughout various seasons on a daily basis (Sometimes even more frequently).
Not all the pictures are usable, some of them just contain dirt and
uninteresting elements. Some, however, contain plants that interest the
biologist. It is these images that make out initial data points.

Before we begin our analysis we need to detect which pictures contain the
relative element that we are measuring. We choose a series of pictures that
capture the phenological stages of an element of interest. 


After choosing the images, we then apply
models to try to generate a signal from the image that will tell us the
phenological stage that the 

\section{Discrete Model}

\section{Continuous Model}


\end{document}

